\documentclass[12pt]{article}
\special{papersize=8.5in,11.0in}
\usepackage{graphicx,hyperref,fullpage}
\usepackage{color}
\usepackage{amsmath}
\usepackage{amssymb}
\pagestyle{empty}
\newcommand{\bfr}{{\bf r}}
\begin{document}

\noindent MA 3475\,\,Exam II Review 


\begin{enumerate}
\item Find the extremal of the functional
$$
J[y,z] = \int_0^1 \left((y^{\prime})^2 - (z^{\prime})^2 - 8y'y -4 y^2 \right)\,dx,
$$

subject to the boundary conditions $y(0) = 1, y(1)=  0, z(0) = 0, z(1) = e$.

{\bf Answer}

Set up and solve the system of Euler-Lagrange equations.

\begin{align*}
0 &= F_y - \frac{d}{dx}F_{y'} = -8y' -8y - \frac{d}{dx}[2y' - 8y] = -8y' - 8y - 2y'' + 8y' = -2y'' - 8y \\
0 &= F_z - \frac{d}{dx}F_{z'} = 0 - \frac{d}{dx}[-2z'] = 2z'' \\
& \\
0 &= -2y'' - 8y \implies 0 = y'' + 4y \\
0 &= r^2 + 4 \\
r &= \pm 2 i \\
y(x) &= c_1\cos(2x) + c_2 \sin(2x) \\
& \\
0 &= 2z'' \implies z(x) = c_3x + c_4
\end{align*}

Apply the boundary conditions.

\begin{align*}
1 &= y(0) = c_1 \\
0 &= y(1) = c_1\cos 2 + c_2 \sin 2 = \cos 2 + c_1 \sin 2 \implies c_2 = -\frac{\cos 2}{\sin 2} \\
& \\
0 &= z(0) = c_4 \\
e &= z(1) = c_3 + c_4 = c_3
\end{align*}

$$\boxed{y(x) = \cos 2x  -\frac{\cos 2}{\sin 2} \sin 2x} $$
$$\boxed{z(x) = ex }$$

\newpage

\item Find the extremal of the functional
$$
J[y,z] = \int_a^b \left((y^{\prime})^2 + (z^{\prime})^2 + y z \right)\,dx
$$

{\bf Answer}

Set up and solve the system of Euler-Lagrange equations.

\begin{align*}
0 &= F_y - \frac{d}{dx}F_{y'} = z - \frac{d}{dx}[2y'] = z - 2y'' \\
0 &= F_z - \frac{d}{dx}F_{z'} = y - \frac{d}{dx}[2z'] = y - 2z''\\
& \\
0 &= z-2y''\\
0 &= y - 2z''\\
& \\
0 &= z - 2y'' \implies 2y^{(iv)} = z''\\
\text{Then, } 0 &= y - 2z'' = y - 2(2y^{(iv)}) = y - 4y^{(iv)}\\
& \\
0 &= 4r^4 - 1 \quad \text{(characteristic equation)}\\
0 &= r^4 - \frac{1}{4} = (r^2 - \frac{1}{2})(r^2 + \frac{1}{2})\\
r &= \pm \frac{\sqrt{2}}{2}, \pm \frac{\sqrt{2}}{2} i
\end{align*}

$$\boxed{y(x) = c_1\exp\left(\frac{\sqrt{2}}{2} x \right)+ c_2 \exp \left(-\frac{\sqrt{2}}{2}\right) + c_3 \cos \left(\frac{\sqrt{2}}{2} x \right) + c_4 \sin \left(\frac{\sqrt{2}}{2}x \right)} $$


We can use a similar process to solve for $z(x)$ (the differential equation is $4z^{(iv)} - z = 0$) or use that $z = 2y''$ to get $z(x)$, which allows us to express $z(x)$ using the same constants used to express $y(x)$.


$$\boxed{z(x) = 2y''(x) =  c_1\exp\left(\frac{\sqrt{2}}{2} x \right)+ c_2 \exp \left(-\frac{\sqrt{2}}{2}\right) - c_3 \cos \left(\frac{\sqrt{2}}{2} x \right) - c_4 \sin \left(\frac{\sqrt{2}}{2}x \right)}$$


\newpage
\item Find the extremal of the functional
%\begin{eqnarray*}
$$
J[y] = \int_0^{\pi/2} \left((y^{\prime})^2 - (y^{\prime\prime})^2 \right) \,dx
$$
subject to the boundary conditions $y(0) = 0, y'(0) = 0, y\left(\frac{\pi}{2}\right) = 1, y'\left(\frac{\pi}{2}\right) = 1$.

{\bf Answer} Set up and solve the Euler-Lagrange equation.

\begin{align*}
0 &= F_y - \frac{d}{dx}F_{y'} + \frac{d^2}{dx^2}F_{y''} \\
&= 0 -\frac{d}{dx}[2y'] + \frac{d^2}{dx^2}[-2y'']\\
&= -2y'' - 2y^{(iv)} \\
& \\
0 &= -2r^2 - 2r^4 \quad \text{(characteristic equation)} \\
0 &= r^2(r^2+1) \\
&\\
r = 0 & \text{ (double root), } r = \pm i \\
&\\
y(x) &= c_0 + c_1x + c_2\cos x + c_3 \sin x \\
y'(x) &= c_1 - c_2 \sin x + c_3 \cos x
\end{align*}

Apply the boundary conditions.

\begin{align*}
0 &= y(0) = c_0 + c_2 \\
0 &= y'(0) = c_1 + c_3 \\
1 &= y\left(\frac{\pi}{2}\right) = c_0 + c_1\frac{\pi}{2} + c_3\\
1 &= y'\left(\frac{\pi}{2}\right) = c_1-c_2
\end{align*}

$$c_0 = 1, c_1 = 0, c_2 = -1, c_3 = 0 $$
$$\boxed{y(x) = 1 - \cos x} $$

\newpage
\item Find the extremal of the functional
%\begin{eqnarray*}
$$
J[y] = \int_a ^b \left((y''')^2  + (y'')^2  \right)\,dx
$$

How many boundary conditions do you need to specify at the two endpoints to determine the extremal uniquely?

{\bf Answer} Set up and solve the Euler-Lagrange equation.

\begin{align*}
0 &= F_y - \frac{d}{dx}F_{y'} + \frac{d^2}{dx^2}F_{y''} - \frac{d^3}{dx^3}F_{y'''} \\
&= 0 - \frac{d}{dx}[0] + \frac{d^2}{dx^2}[2y''] - \frac{d^3}{dx^3}[2y'''] \\
&= 2y^{(iv)} - 2y^{(vi)} \\
& \\
0 &= 2r^4 - 2r^6 \quad \text{(characteristic equation)} \\
0 &= r^4(r^2-1) \implies r = 0 \text{ (root with multiplicity 4) or } r= \pm 1 
\end{align*}

$$\boxed{ y(x) = c_0 + c_1x +c_2x^2 + c_3x^3 + c_4e^x + c_5e^{-x}}$$

You need to have 6 boundary conditions specified in order to determine the extremal uniquely.

\newpage

\item Find the extremal of the functional
$$
J[y] = \int_0^1 \left( x^2 + (y^{\prime})^2 \right) \,dx
$$
subject to the conditions $y(0) = 0, y(1) = 0$ and the constraint $\int_0^1 \left( x^2 + 2 y(x) \right)\,dx = 1$.

{\bf Answer}

Let $G(x,y,y') = x^2 + 2y(x)$ so that $1 = \int_0^1 \left( x^2 +2y(x)\right) \; dx = \int_0^1 G(x,y,y') \; dx$. First set up and solve the Euler-Lagrange equation for $y(x)$. 

\begin{align*}
0 &= F_y - \frac{d}{dx}F_{y'} + \lambda (G_y - \frac{d}{dx}G_{y'}) \\
&= 0 - \frac{d}{dx}[2y'] + \lambda (2 - \frac{d}{dx}[0]) \\
&= -2y''  + 2\lambda \\
y''(x) &= \lambda \\
y(x) &= \frac{\lambda}{2}x^2 + Ax + B \\
\end{align*}

Next apply the boundary conditions.

\begin{align*}
0 &= y(0) = B \\
0 &= y(1) = \frac{\lambda}{2} + A \implies A = -\frac{\lambda}{2} \\
y(x) &= \frac{\lambda}{2}x^2 - \frac{\lambda}{2}x
\end{align*}

Finally, apply the integral constraint.

\begin{align*}
1 &= \int_0^1 \left(x^2 + \lambda x^2 - \lambda x \right) \, dx \\
&= \frac{1}{3} + \frac{\lambda}{3} - \frac{\lambda}{2} = \frac{1}{3} -\frac{\lambda}{6} \\
\lambda &= -4
\end{align*}

$$\boxed{ y(x) = -2x^2+2x } $$

\newpage

\item Find the extremal of the functional
$$
J[y] = \int_0^1 \frac{1}{2} (y^{\prime})^2 \,dx
$$
subject to the conditions $y(0) = 0, y(1) = 1$ and the constraint $\int_0^1 \frac{1}{2}y^2 \,dx = 1$. For this problem   you may stop after expressing $y(x)$ in terms of the multiplier $\lambda$ but this should be the only unknown constant. For a challenge, determine $\lambda$ to finish solving the problem.

{\bf Answer:} 

\begin{align*}
0 &= 0 - \frac{d}{dx}[y'] + \lambda y - \frac{d}{dx}[0] \\
0 &= y'' - \lambda y \\
&\\
0 &= r^2 - \lambda \\
r &= \pm \sqrt{\lambda} \\
& \\
y(x) &= c_1e^{\sqrt{\lambda}x} + c_2 e^{-\sqrt{\lambda}x} \\
\end{align*}

Apply the boundary conditions.

\begin{align*}
0 &= y(0) = c_1 + c_2 \implies c_2 = -c_1\\
1 &= y(1) = c_1 \left(e^{\sqrt{\lambda}} -e^{-\sqrt{\lambda}} \right) = 2c_1 \sinh \sqrt{\lambda} \implies c_1 = \frac{1}{2\sinh \sqrt{ \lambda}} \\
y(x) &= \frac{1}{2\sinh \sqrt{ \lambda}}\left(e^{\sqrt{\lambda} x} -e^{-\sqrt{\lambda} x} \right) = \frac{2\sinh \sqrt{\lambda}x }{2\sinh \sqrt{ \lambda}} = \frac{\sinh \sqrt{ \lambda}x }{\sinh \sqrt{ \lambda}}
\end{align*}

Apply the integral constraint to determine $\lambda$. 

\begin{align*}
1 &= \int_0^1 \frac{1}{2}\left(\frac{\sinh \sqrt{ \lambda}x }{\sinh \sqrt{ \lambda}}\right)^2 \, dx 
\end{align*}
\end{enumerate}
\end{document}
